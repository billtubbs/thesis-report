% !TEX encoding = UTF-8 Unicode
\chapter*{Abstract}             % ne pas numéroter
\label{chap-abstract}           % étiquette pour renvois
\phantomsection\addcontentsline{toc}{chapter}{\nameref{chap-abstract}} % inclure dans TdM

\begin{otherlanguage*}{english}
  % Abstract from IFAC paper
 \textbf{ INCOMPLETE: This is the abstract from IFAC report -- for starters.}
  
  Changes in ore properties create challenges for the control and optimization of comminution operations because they are generally difficult to measure in real time and have significant impacts on the grinding process and downstream operations. The effectiveness of a multi-model observer to detect and estimate step changes in the particle size distribution of the ore feed to a semi-autogenous grinding (SAG) mill using noisy measurements of the product particle size, is evaluated using a simulation model of the process. The observer maintains multiple hypotheses about the disturbance until their likelihood given the measurements can be determined and used to estimate the disturbance and the true process output. The results demonstrate that the multi-model observer has lower overall estimation errors than a single Kalman filter because it responds to changes in the output quickly without a compromised sensitivity to noise during steady-state. Real-time estimation of changes in ore feed properties in grinding operations could have significant benefits, however, more work is needed to characterize these disturbances, to determine if the process and disturbance models can be identified in practice, and to estimate the potential benefits.
\end{otherlanguage*}
