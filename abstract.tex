% !TEX encoding = UTF-8 Unicode
\chapter*{Abstract}             % ne pas numéroter
\label{chap-abstract}           % étiquette pour renvois
\phantomsection\addcontentsline{toc}{chapter}{\nameref{chap-abstract}} % inclure dans TdM

\begin{otherlanguage*}{english}
  
Changes in ore properties create challenges for the control of semi-autogenous grinding (\acrshort{SAG}) mills because they are generally difficult to measure in real time and have significant impacts on the grinding process. Although there is a lack of understanding of the nature of variations in ore properties in real operations, available data on the particle size distribution indicates they are characterised by abrupt step changes and ramp behaviours, which standard disturbance models used in process control are not designed for. \hlep{In this work, an alternative disturbance model known as the randomly-occurring deterministic disturbance ({\acrshort{RODD}}) is considered. This has a switching random noise input, which makes it suitable for modelling these types of disturbances. However, since the noise is non-Gaussian, a standard Kalman filter, which is typically used for state estimation, is not optimal. The capabilities of two multiple-model observers to detect and estimate the states of systems subjected to unmeasured {\acrshort{RODD}}s are evaluated. These observers maintain multiple estimates of the system states based on different hypotheses about the switching of the disturbance. The likelihood of each hypothesis given the available measurements is estimated and used to produce a better, although still sub-optimal, estimate the process states and output.

Two types of sub-optimal multiple-model observer are evaluated and compared to a standard Kalman filter using simulated noisy measurements from three different process systems---a linear system with one {\acrshort{RODD}} and one output, a linear system with two {\acrshort{RODD}}s and two-outputs, and a realistic grinding process simulation with a switching ore feed and one output measurement. The results show that the multiple-model observers detect and respond quickly to step changes in the disturbance, without having a compromised sensitivity to noise during steady-state. This suggests that more realistic models of ore feed disturbances and improved real-time estimation of changes in ore properties could have benefits in terms of improved process control, although the improvement compared to a single Kalman filter was found to depend on the magnitude of the measurement noise.}
% Removed as suggested by Eric
%  More realistic models of ore feed disturbances and improved real-time estimation of changes in ore properties could have significant benefits in terms of improved control and reduced variation in process variables. However, more work is needed to characterize real disturbances, to determine if the disturbance models can be identified in practice, and to estimate the potential benefits in process control performance.
\end{otherlanguage*}
