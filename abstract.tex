% !TEX encoding = UTF-8 Unicode
\chapter*{Abstract}             % ne pas numéroter
\label{chap-abstract}           % étiquette pour renvois
\phantomsection\addcontentsline{toc}{chapter}{\nameref{chap-abstract}} % inclure dans TdM

\begin{otherlanguage*}{english}
  
  Changes in ore properties create challenges for the control of semi-autogenous grinding (SAG) mills because they are generally difficult to measure in real time and have significant impacts on the grinding process. Although there is a lack of understanding of the nature of variations in ore properties in real operations, available data on the particle size distribution indicates they are characterised by abrupt step changes and ramp behaviours, which standard disturbance models used in process control are not designed for. In this work, the capabilities of two multiple-model observers to detect and estimate randomly-occurring deterministic disturbances (\acrshort{RODD}s) are evaluated using numerical simulations, including a realistic grinding process simulation with a switching ore feed. The observers detect and respond quickly to step changes in the disturbance, without having a compromised sensitivity to noise during steady-state. The simulation results demonstrate that the multiple-model observers have advantages over a standard Kalman filter, which is typically used for state estimation in process control applications. 
  
  More realistic models of ore feed disturbances and improved real-time estimation of changes in ore properties could have significant benefits in terms of improved control and reduced variation in process variables. However, more work is needed to characterize real disturbances, to determine if the disturbance models can be identified in practice, and to estimate the potential benefits in process control performance.
\end{otherlanguage*}
