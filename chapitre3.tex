\chapter{Disturbance Model Identification}
\label{chap-identification}

\section{Identifying disturbance models}

t.b.d. —depends if we identify a useful method.

\begin{itemize}
	\item Test methods of identifying the disturbance model parameters from simulated measurements.
	\item E.g. Variational inference? Sequential inference techniques?  Need help with this.
	\item Using data from linear system from section 2.1.
	\item Using data from grinding model simulations from section 2.3.
\end{itemize}

\section{Evaluation with real industrial data (t.b.d)}

t.b.d. —depends if the data is good enough for the purpose.

\begin{itemize}
	\item Description of plant and data acquisition process.
	\item Preparation and pre-processing of data—outliers, missing values, standardization parameters.
	\item Ideally: Demonstrate disturbance characterisation (model structure), identification (parameters) with one or more method and discuss results. 
\end{itemize}