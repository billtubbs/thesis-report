\chapter{Disturbance Model Identification}
\label{chap-identification}

\section{Identifying disturbance models}

t.b.d. —depends if we identify useful methods.

Outline notes:
\begin{outline}
	\1 Test methods of identifying the disturbance model parameters from simulated measurements.
	\2 Using data from linear system from section 2.1.
	\2 Using data from grinding model simulations from section 2.3.
	\1 Results of identifying the random shock distribution with noise using Expectation-Maximization (EM) algorithm to fit Gaussian Mixture models.
	\1 Problem of inference of the RODD - simply inverting model magnifies noise.
	\1 Eriksson and Isaksson's approach to system detection. Using an observer to identify system from a set of known systems.
	\1 Demos of other methods:
	  \2 Total variation de-noising (for step disturbance signals)
	  \2 Variational inference (how is this different from EM?)
	\1 Sequential inference techniques? E.g. Sequential Monte Carlo. Don't know enough about this. Mention in conclusion and recommendations for future work.
\end{outline}

\section{Characterizing ore disturbances from operating data}

In this section, some results are presented of analysing the ore data sample obtained from an operating mine, The complete time series consisted of 1915 hourly values of the two ore property indicators, which are described in the last sub-section of the Introduction of this report.



t.b.d. —depends if the data is good enough for the purpose.

Outline notes:
\begin{itemize}
	\item Description of plant and data acquisition process.
	\item Preparation and pre-processing of data—outliers, missing values, standardization parameters.
	\item Ideally: Demonstrate disturbance characterisation (model structure), identification (parameters) with one or more method and discuss results. 
\end{itemize}