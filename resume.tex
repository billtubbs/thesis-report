% !TEX encoding = UTF-8 Unicode
\chapter*{Résumé}               % ne pas numéroter
\label{chap-resume}             % étiquette pour renvois
\phantomsection\addcontentsline{toc}{chapter}{\nameref{chap-resume}} % inclure dans TdM

\begin{otherlanguage*}{french}
% With help from Alex
Les changements dans les propriétés du minerai apportent des défis pour le contrôle des broyeurs semi-autogènes (\acrshort{SAG}) car ils sont généralement difficiles à mesurer en temps réel et ont des impacts significatifs sur le procédé. Bien qu'il y ait un manque de compréhension de la nature des variations des propriétés du minerai dans les opérations réelles, les données disponibles sur la distribution granulométrique indiquent qu'elles sont caractérisées par des changements abrupts et des comportements en rampe, pour lesquels les modèles de perturbation standard utilisés dans le contrôle des procédés ne sont pas conçus. \hlep{Dans ce travail, un modèle de perturbation alternatif connu sous le nom de perturbation déterministe se produisant de manière aléatoire (\textit{randomly-occurring deterministic disturbances} ({\acrshort{RODD}}s)) est considéré. Celui-ci possède une entrée de bruit aléatoire de commutation, ce qui le rend adapté à la modélisation de ces types de perturbations. Cependant, étant donné que le bruit n'est pas gaussien, un filtre de Kalman standard, qui est généralement utilisé pour l'estimation d'état, n'est pas optimal. Les capacités de deux observateurs à modèles multiples capables de détecter et d'estimer les états de systèmes soumis à des {\acrshort{RODD}} non mesurés sont évaluées. Ces observateurs maintiennent plusieurs estimations des états du système sur la base de différentes hypothèses sur la commutation de la perturbation. La vraisemblance de chaque hypothèse compte tenu des mesures disponibles est estimée et utilisée pour produire une meilleure estimation, bien que toujours sous-optimale, des états et de la sortie du procéde.}

 \hlep{Deux types d'observateurs à modèles multiples sous-optimaux sont évalués et comparés à un filtre de Kalman standard en utilisant des mesures de bruit simulées à partir de trois systèmes de procédes différents---un système linéaire avec un {\acrshort{RODD}} et une sortie, un système linéaire avec deux {\acrshort{RODD}} et deux sorties, et une simulation réaliste du procéde de broyage avec une alimentation en minerai de commutation et une mesure de sortie. Les résultats montrent que les observateurs à modèles multiples détectent et réagissent rapidement aux changements progressifs de la perturbation, sans avoir une sensibilité compromise au bruit en régime permanent. Cela suggère que des modèles plus réalistes de perturbations du minerai alimenté et une meilleure estimation en temps réel des changements dans les propriétés du minerai pourraient avoir des avantages en termes d'amélioration du contrôle du procéde, bien que l'amélioration par rapport à un seul filtre de Kalman dépende du niveau du bruit de mesure.}
% Removed as suggested by Eric
%Des modèles plus réalistes des perturbations des propriétés du minerai alimenté pourraient avoir des avantages significatifs en termes de contrôle amélioré et de variabilité réduite des variables de procédés. Cependant, des travaux supplémentaires sont nécessaires pour caractériser les perturbations réelles, pour déterminer si les modèles de perturbation peuvent être identifiés en pratique et pour estimer les avantages potentiels dans les performances de contrôle des procédés.
\end{otherlanguage*}
