% !TEX encoding = UTF-8 Unicode
\chapter*{Résumé}               % ne pas numéroter
\label{chap-resume}             % étiquette pour renvois
\phantomsection\addcontentsline{toc}{chapter}{\nameref{chap-resume}} % inclure dans TdM

\begin{otherlanguage*}{french}
Les changements dans les propriétés du minerai créent des défis pour le contrôle des broyeurs semi-autogènes (SAG) car ils sont généralement difficiles à mesurer en temps réel et ont des impacts significatifs sur le procédé. Bien qu'il y ait un manque de compréhension de la nature des variations des propriétés du minerai dans les opérations réelles, les données disponibles sur la distribution granulométrique indiquent qu'elles sont caractérisées par des changements abrupts et des comportements en rampe, pour lesquels les modèles de perturbation standard utilisés dans le contrôle des procédés ne sont pas conçus. Dans ce travail, les capacités de deux observateurs à modèles multiples à détecter et à estimer \textit{randomly-occurring deterministic disturbances} (\gls{RODD}s) dans la taille de l'alimentation en minerai sont évaluées à l'aide de simulations numériques, y compris une simulation réaliste du procédé de broyage avec une alimentation en minerai de commutation. Les observateurs détectent et réagissent rapidement aux changements progressifs de la perturbation, sans compromettre la sensibilité au bruit en régime permanent. Les résultats de la simulation démontrent que les observateurs à modèles multiples présentent des avantages par rapport à un filtre de Kalman standard, qui est généralement utilisé pour l'estimation d'état dans les applications de contrôle de procédés.

Des modèles plus réalistes des perturbations de l'alimentation en minerai et une meilleure estimation en temps réel des changements dans les propriétés du minerai pourraient avoir des avantages significatifs en termes de contrôle amélioré et de variation réduite des variables de procédés. Cependant, des travaux supplémentaires sont nécessaires pour caractériser les perturbations réelles, pour déterminer si les modèles de perturbation peuvent être identifiés dans la pratique, et pour estimer les avantages potentiels dans les performances de contrôle des procédés.
\end{otherlanguage*}
