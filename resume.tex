% !TEX encoding = UTF-8 Unicode
\chapter*{Résumé}               % ne pas numéroter
\label{chap-resume}             % étiquette pour renvois
\phantomsection\addcontentsline{toc}{chapter}{\nameref{chap-resume}} % inclure dans TdM

\begin{otherlanguage*}{french}
% With help from Alex
Les changements dans les propriétés du minerai apportent des défis pour le contrôle des broyeurs semi-autogènes (\acrshort{SAG}) car ils sont généralement difficiles à mesurer en temps réel et ont des impacts significatifs sur le procédé. Bien qu'il y ait un manque de compréhension de la nature des variations des propriétés du minerai dans les opérations réelles, les données disponibles sur la distribution granulométrique indiquent qu'elles sont caractérisées par des changements abrupts et des comportements en rampe, pour lesquels les modèles de perturbation standard utilisés dans le contrôle des procédés ne sont pas conçus.

Dans ce travail, un modèle de perturbation déterministe se produisant de manière aléatoire (\textit{randomly-occurring deterministic disturbances} ({\acrshort{RODD}}s)) est considéré. Celui-ci possède une entrée commutant entre deux bruits aléatoires. Puisque la perturbation n’est pas gaussienne, un filtre de Kalman standard, qui est généralement utilisé pour l’estimation d’état, n’est pas optimal. Les capacités de deux observateurs à modèles multiples de détecter et d’estimer les états de systèmes soumis à des RODD non mesurés sont évaluées. Ces observateurs maintiennent plusieurs estimations des états du système sur la base de différentes hypothèses sur la commutation de la perturbation. La vraisemblance de chaque hypothèse compte tenu des mesures disponibles est évaluée et utilisée pour produire une meilleure estimation des états et de la sortie du procédé, qui demeure toutefois sous-optimale.

Deux types d’observateurs à modèles multiples sous-optimaux sont évalués et comparés à un filtre de Kalman standard en utilisant des mesures de bruit simulées à partir de trois systèmes différents—un système linéaire avec un RODD et une sortie, un système linéaire avec deux RODD et deux sorties, et une simulation réaliste d’un circuit de broyage avec une mesure de sortie et une alimentation commutant entre deux types de minerai.

Les résultats montrent que les observateurs à modèles multiples détectent et réagissent rapidement aux changements instantanés de la perturbation, sans pour autant avoir une sensibilité accrue au bruit lorsqu’en régime permanent.  Cela suggère que des modèles plus réalistes de perturbations du minerai alimenté et une meilleure estimation en temps réel des changements dans les propriétés du minerai pourraient améliorer le contrôle du procédé, bien que les gains par rapport à un filtre unique de Kalman dépendent du niveau du bruit de mesure.

% Removed as suggested by Eric
%Des modèles plus réalistes des perturbations des propriétés du minerai alimenté pourraient avoir des avantages significatifs en termes de contrôle amélioré et de variabilité réduite des variables de procédés. Cependant, des travaux supplémentaires sont nécessaires pour caractériser les perturbations réelles, pour déterminer si les modèles de perturbation peuvent être identifiés en pratique et pour estimer les avantages potentiels dans les performances de contrôle des procédés.
\end{otherlanguage*}
