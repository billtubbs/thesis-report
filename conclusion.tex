% !TEX encoding = UTF-8 Unicode
\chapter*{Conclusion}           % ne pas numéroter
\label{chap-conclusion}         % étiquette pour renvois
\phantomsection\addcontentsline{toc}{chapter}{\nameref{chap-conclusion}} % inclure dans TdM

\textbf{NOT COMPLETE}

\begin{itemize}
	\item Summarize results
	\item Summarize issues/drawbacks and potential ways to mitigate them.
	\item Identify gaps and suggest areas of further research/data collection.
	\item Reiterate potential benefits and need to quantify these.
\end{itemize}

% Text from IFAC paper
While the multi-model observer has a number of clear advantages over a single Kalman filter, its limitations are important to recognize. Firstly, the higher variance of the noise model during the transitions, which is needed for a fast response, also increases the variance of the estimates during these periods, resulting in larger errors---46\% higher than those of KF3 during transition periods.

Secondly, the multiple-model algorithm is limited to infrequently-occurring disturbances where the disturbance model is known. Finally, the complexity of the algorithm is a significant disadvantage from the perspective of practical implementation.


The settling time of the process in number-of-sample-periods is the settling time divided by the sampling interval,
\begin{equation} \label{eq:Nsettle}
	\glsadd{Nset95}N_{\pm 5\%}=\frac{t_{\pm 5\%}}{T_s}.
\end{equation}

\begin{table}[ht]
	\begin{center}
		\caption{Summary of experiments -- observer parameters} \label{tb:summary-all-sims}
		% See: https://texblog.org/2019/06/03/control-the-width-of-table-columns-tabular-in-latex/
		\begin{tabular}{
				>{\centering\arraybackslash}p{0.18in}
				>{\centering\arraybackslash}p{0.52in}
				>{\centering\arraybackslash}p{0.43in}
				>{\centering\arraybackslash}p{0.26in}
				>{\centering\arraybackslash}p{0.26in}
				>{\centering\arraybackslash}p{0.28in}
				>{\centering\arraybackslash}p{0.8in}
				>{\centering\arraybackslash}p{0.8in}
				>{\centering\arraybackslash}p{0.8in}}
			Sec. & System & Dim. & \gls{Ts} & \gls{SNR} & $N_{\pm 5\%} $ & MKF--SF95 (\gls{nf},\gls{m},\gls{d}) & MKF--SF1 (\gls{nf},\gls{m},\gls{d}) & MKF-SP  (\gls{nh},\gls{nmin}) \\
			\hline
			\ref{sec:sim-obs-lin} & linear & \gls{SISO} & 0.5 & 10 & 9 & 5,1,1 & 6,1,2 & 10,7 \\
			\ref{sec:sim-obs-lin} & linear & $2 \times 2$ & 1 & 5 & 26 & 15,2,3 & 18,2,5 & 46,21 \\
			\ref{sec:sim-ore-SISO} & nonlinear & \gls{SISO} & 0.05 & 1.9 & 24 & 60,2,10 & 60,2,12 & 25,23  \\
			\hline
		\end{tabular}
	\end{center}
\end{table}

These are quite different to those of the observers evaluated in section \ref{sec:sim-obs-lin}. In the case of the sequence fusion observers, the best lengths of the detection horizon are much longer at 10 and 12 sample periods, compared to 1 and 2 for the SISO linear system and 3 and 5 for the MIMO linear system. The best fusion horizons here are also much longer, at 60 sample periods compared to 5, 6, and 15 for linear system simulations. In the case of the sequence pruning observer, the \gls{nmin} parameter is higher, at 23, compared to 7 for the SISO linear system, but similar to its value for the MIMO linear system which was 21. Although the numbers of hypotheses used by all three observers here are higher than for the SISO linear system, they are significantly lower than the numbers needed by the observers for the MIMO linear system, which were 116, 58, and 46.

