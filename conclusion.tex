% !TEX encoding = UTF-8 Unicode
\chapter*{Conclusion}           % ne pas numéroter
\label{chap-conclusion}         % étiquette pour renvois
\phantomsection\addcontentsline{toc}{chapter}{\nameref{chap-conclusion}} % inclure dans TdM

\textbf{NOT COMPLETE}

\begin{itemize}
	\item Summarize results
	\item Summarize issues/drawbacks and potential ways to mitigate them.
	\item Identify gaps and suggest areas of further research/data collection.
	\item Reiterate potential benefits and need to quantify these.
\end{itemize}

% Text from IFAC paper
While the multi-model observer has a number of clear advantages over a single Kalman filter, its limitations are important to recognize. Firstly, the higher variance of the noise model during the transitions, which is needed for a fast response, also increases the variance of the estimates during these periods, resulting in larger errors---46\% higher than those of KF3 during transition periods.

Secondly, the multiple-model algorithm is limited to infrequently-occurring disturbances where the disturbance model is known. Finally, the complexity of the algorithm is a significant disadvantage from the perspective of practical implementation.




