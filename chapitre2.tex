\chapter{Simulation results}
\label{chap-simulation}

\section{Estimating RODD disturbances}

\begin{itemize}
	\item RODDs are easy to generate.
	\item Show example plots of various types of RODD disturbance described in chapter 1.
	\item Describe SISO or 2x2 MIMO system with RODD step disturbance(s) at input
	\item Tuning of different single Kalman filters
	\item Tuning of Robertson and Eriksson and Isaksson sub-optimal methods
	\item Comparison simulation results and metrics
	\item Compare and contrast
	\item Conclude on pros, cons of each
\end{itemize}

\section{Control performance}

\begin{itemize}
	\item Use best observer from previous section.
	\item Same system in closed loop with LQI/LQRcontroller.
	\item Performance metrics — e.g. tracking error.
	\item Robustness?  E.g. stability margins.
\end{itemize}


\section{Grinding model simulations}

\begin{itemize}
	\item Non-linear model
	\item Describe simulations with grinding simulation model (IFAC paper)
	\item Include sensitivity analysis - model errors, observer (RODD) parameters.
	\item Discuss applications and potential benefits (e.g. RTO).
\end{itemize}

\section{Disturbance model identification (t.b.d)}

t.b.d. —depends if we identify a method.

\begin{itemize}
	\item Test methods of identifying the disturbance model parameters from measurements.
	\item Using data from linear system from section 2.1.
	\item Using grinding model simulations from section 2.3 
\end{itemize}

\section{Evaluation with real industrial data (t.b.d)}

t.b.d. —depends if the data is good enough for the purpose.
