\chapter{<Titre de l'annexe>}     % numérotée
\label{chap-}                   % étiquette pour renvois (à compléter!)

\section{Multiple model observer algorithm}

\begin{itemize}
	\item Multi-model algorithm.
\end{itemize}


\section{Sensitivity analyses}

\subsection{Pseudo-random numbers}

Many of the simulation results are sensitive to the initialization of the pseudo-random number generator (PRNG) used to simulate random processes. In particular, the RODD step disturbance (\ref{eq:wpk2}) is simulated by generating three pseudo-random sequences, two random noise sequences and a random binary sequence to simulate the infrequent shocks.  Since the random shocks are infrequent and tend to have a large magnitude, their effect on the simulation results can be significant.

To visualize this effect, consider the plot in Figure \ref{fig:rod-obs-sim-1-3KF-seed-crmse-statsplot}. This shows the RMSE of the output estimates of the three Kalman filters described in Section \ref{sim-obs-lin-1} for 10 simulations, each generated with a different \textit{seed}—the seed is a scalar argument used to initialize the PRNG algorithm in a unique state. The RMSE is calculated for every simulation duration, $t_N=0.5,1,1.5,...,2500$. The coloured areas represent the range between the lowest and the highest RMSE obtained for the 10 different simulations of each duration. The dark lines represent the median values.

\begin{figure}[htp]
	\centering
	\includegraphics[width=14cm]{images/rod_obs_sim1_3KF_seed_crmse_statsplot.pdf}
	\caption{Effect of random variables on the RMSE results.}
	\label{fig:rod-obs-sim-1-3KF-seed-crmse-statsplot}
\end{figure}

As expected, the differences in the results due to PRNG initialization are smaller the greater the length of the simulation. However, the magnitude of the differences is not the same for each observer. The RMSE values of KF1, which has the lowest gain, are more sensitive to the random initialization than the other two filters. Therefore it is not possible to estimate the expected value of the RMSE of KF1 from a single simulation of this duration---a longer simulation or a larger number of simulations would be needed. On the other hand, the variations in the RMSEs of KF2 and KF3 are significantly lower. After the full length of the simulations ($t_N=2500$), the RMSE of KF2 is between -0.0039 (-2.5\%) and 0.0019 (1.3\%) of the median value, which is $0.1550$.  That of KF3 is between -0.0035 (-4.0\%) / 0.0047 (5.3\%) of the median, 0.0889.  Therefore it is reasonable to conclude that the RMSE of KF3 is consistently about 0.05 lower than that of KF2. Note that the RMSEs of KF2 and KF3 reported in Section \ref{sim-obs-lin-1} are \alert{0.XXXX} and \alert{0.XXXX}. These values are both within the minimum and maximum values from this sensitivity analysis (in fact, the simulation output used to produce the main results is one of the 10 shown here).

